%!TEX TS-program = xelatex
\documentclass[]{friggeri-cv}
\addbibresource{bibliography.bib}
\hypersetup{colorlinks=false,allbordercolors=white}
\begin{document}

\header{Bram }{De Jaegher}
       {Bioscience engineer}{Curriculum}{Vitae}
    % In the aside, each new line forces a line break
\begin{aside}
  \section{about}
    Abelendreef 21
    8300 Knokke-Heist
    Belgium
    ~
    \href{mailto:bram.de.jaegher@gmail.com}{bram.de.jaegher@gmail.com}
    \href{http://be.linkedin.com/in/BramDeJaegher}{LinkedIn}
    \href{https://github.com/Beramos}{Github}
    ~
    Driving licence: B \vspace{-0.8mm}
  \section{languages}
    Dutch: native language
    English: C2 (CEFR)
    French:  B1  (CEFR) B \vspace{3.3mm}
  \section{additional skills}
  Computational fluid dynamics
  Classical control theory 
  Modern control theory 
  Machine learning
  MS office \vspace{8mm}
  \section{programming} 
  \textit{Working knowledge}
  MATLAB/Simulink
  Python 2.7/3
  LaTeX
  R \vspace{0.1cm}
  \textit{Basic knowledge}
  OpenFOAM (CFD)
  HTML5/CSS
  C++
\end{aside}
%\newgeometry{left=5cm}
\section{education}
\begin{entrylist}
    \entry
    {2014–2016}
    {M.Sc. (In progress)}
    {Ghent university}
    {Bioscience engineering\\
    \emph{Chemistry and foodtechnology}}
  \entry
    {2011-2014}
    {B.Sc. cum laude}
    {Ghent university}
    {Bioscience engineering\\
    \emph{Chemistry and Bioproces technology}}
  \entry
    {2005-2011}
    {GSCE in Math and Sciences}
    {Royal Atheneum Knokke-Heist}
    {4.3/5 GPA}
   
%\restoregeometry
\end{entrylist}
\section{experience}

\begin{entrylist}
  \entry
    {08-09/2014}
    {Research internship}
    {S\~{a}o Carlos Institute of Physics - University of S\~{a}o Paulo}
    {Computer vision techniques for polymer recognition\\ on atomic-force microscopy images}
\end{entrylist}

\section{scriptions}

\begin{entrylist}
  \entry
    {2016}
    {Master's thesis}
    {Ghent university}
    {Spatio temporal modelling of filter cake formation\\ in membrane bioreactors}
  \entry
    {2014}
    {Bachelor's thesis}
    {Ghent university}
    {Innovative applications of artificial intelligence\\ in the food industry}
\end{entrylist}

\section{voluntary work}
\begin{entrylist}
 \entry
    {2009-2014}
    {Leader youth movement}
    {Knokke-Heist}
    {102\textsuperscript{e} FOS De Albatros}
\end{entrylist} 
\section{projects}
\begin{entrylist}
 \entry
 {2016}
 {Open Webslides}
 {}
 {Open-source platform for interactive presentation slides \\ \emph{UGent innoversity challenge finals}}
 \entry
 {2016}
 {COCOON: communication \& co-creation online}
 {Ghent university}
 {Education innovation projects 2016}
%  \begin{tabular*}{\textwidth}{l r}
%   \textbf{Open Webslides} \\
%     Open-source platform for interactive presentation slides \\ \emph{UGent innoversity challenge finals} & % \href{http://openwebslides.github.io/}{Open Webslides Homepage}
%   \end{tabular*} \\ \\
% 
%  \begin{tabular*}{\textwidth}{l r}
%   \textbf{COCOON: communication \& co-creation online} \\
%    Education innovation projects 2016 \emph{(Ghent university)} %\href{http://openwebslides.github.io/}{Open Webslides Homepage}
%   \end{tabular*} \\ \\
 
   
%\section{interests}
\end{entrylist} 

%%%%%%%%%%%%%%%%%%%%%%%%%%%%%%%%%%%%%%%%%%%%%%%%%%%%%%%%%%%%%%%%%%%%%%%%%%%%%%%%%%%%%%%%%%%%%%%%%%%%%%%%%%%%%%%%%%%%
%%% This piece of code has been commented by Karol Kozioł due to biblatex errors. 
% 
%\printbibsection{article}{article in peer-reviewed journal}
%\begin{refsection}
%  \nocite{*}
%  \printbibliography[sorting=chronological, type=inproceedings, title={international peer-reviewed conferences/proceedings}, notkeyword={france}, heading=subbibliography]
%\end{refsection}
%\begin{refsection}
%  \nocite{*}
%  \printbibliography[sorting=chronological, type=inproceedings, title={local peer-reviewed conferences/proceedings}, keyword={france}, heading=subbibliography]
%\end{refsection}
%\printbibsection{misc}{other publications}
%\printbibsection{report}{research reports}
%%%%%%%%%%%%%%%%%%%%%%%%%%%%%%%%%%%%%%%%%%%%%%%%%%%%%%%%%%%%%%%%%%%%%%%%%%%%%%%%%%%%%%%%%%%%%%%%%%%%%%%%%%%%%%%%%%%%

%hyperlink images
\makeatletter
\def\parsecomma#1,#2\endparsecomma{\def\page@x{#1}\def\page@y{#2}}
\tikzdeclarecoordinatesystem{page}{
    \parsecomma#1\endparsecomma
    \pgfpointanchor{current page}{north east}
    % Save the upper right corner
    \pgf@xc=\pgf@x%
    \pgf@yc=\pgf@y%
    % save the lower left corner
    \pgfpointanchor{current page}{south west}
    \pgf@xb=\pgf@x%
    \pgf@yb=\pgf@y%
    % Transform to the correct placement
    \pgfmathparse{(\pgf@xc-\pgf@xb)/2.*\page@x+(\pgf@xc+\pgf@xb)/2.}
    \expandafter\pgf@x\expandafter=\pgfmathresult pt
    \pgfmathparse{(\pgf@yc-\pgf@yb)/2.*\page@y+(\pgf@yc+\pgf@yb)/2.}
    \expandafter\pgf@y\expandafter=\pgfmathresult pt
}
\makeatother
\begin{tikzpicture}[remember picture,overlay,every node/.style={anchor=center}]
  %\node at (page cs:0.93,0.58) {\def\svgwidth{0.7cm}\input{linkIcon.pdf_tex}};
  \node at (page cs:0.93,0.48) {\def\svgwidth{0.6cm}\input{linkIcon.pdf_tex}};
  \node at (page cs:0.93,0.02) {\def\svgwidth{0.6cm}\input{linkIcon.pdf_tex}};
  %\node at (page cs:0.93,-0.02) {\def\svgwidth{0.7cm}\input{linkIcon.pdf_tex}};
  \node at (page cs:0.93,-.48) {\def\svgwidth{0.6cm}\input{linkIcon.pdf_tex}};
  \node at (page cs:0.93,-0.58) {\def\svgwidth{0.6cm}\input{linkIcon.pdf_tex}};
  
\end{tikzpicture}

\textbf{Keywords:} mathematical modelling, process control, computational fluid dynamics, optimisation, chemistry, problem solving

\end{document}
%leeftijd geboortejaar
