%!TEX TS-program = xelatex
\documentclass[]{friggeri-cv}
%\addbibresource{bib/bibliography.bib}
\hypersetup{    colorlinks=true,
    linkcolor=black,
    citecolor=black,
    filecolor=black,
    urlcolor=black,}
\begin{document}

\header{Nancy}{Deceuninck}
       {Leerkracht Engels en wiskunde}{Curriculum}{Vitae}
 \begin{aside}
  \vspace{0.6cm}\section{personalia}
    5 okt. '90
    9000 Gent
    Belgi\"e\vspace{0.21cm}
    \linkBlock{\href{mailto:deceunincknancy@gmail.com}{deceunincknancy@gmail.com}}\vspace{0.21cm}
    Driver's licence: B \vspace{-1.25mm}
  \section{talen}
    Nederlands: moedertaal
    Engels: C2
    Frans:  B1  
    Spaans: A2\vspace{3.3mm}
  \section{interesses}
  organiseren
  windsurfen
  reizen
  lezen
  yoga\vspace{3.3mm}
  \section{extra opleidingen}
  The MA Teaching Assistant Programme
  {\footnotesize\addfontfeature{Color=lightgray}Universiteit Gent} \vspace{0.25cm}
  Opleiding tot reisbegeleider
  {\footnotesize\addfontfeature{Color=lightgray}KrisKras VZW} \vspace{3.3mm}
\end{aside}
%{\textbf{keywords:} mathematical modelling, process control, computational fluid dynamics,\\ machine learning, optimisation, chemistry, problem solving}
%\vspace{3mm}
\vspace{-0.4cm}
\section{studies}
\vspace{-0.2cm}
\begin{entrylist}
    \entry
    {2014–2016}
    {Master of Arts in de Taal- en Letterkunde}
    {Universiteit Gent}
    {\emph{afstudeerrichting Engels} \\ \textit{Masterscriptie: “The influence of Emigration on Trauma in \\
    Two Post-Soviet Emigr\'e Novels: Akhtiorskaya’s \\
    ‘Panic in a Suitcase’ and Kalman’s ‘The Cosmopolitans’.”}}
  \entry
    {2013-2015}
    {Schakelprogramma Master in de Taal- en Letterkunde}
    {Universiteit Gent}
    {\emph{afstudeerrichting Engels} \\ \textit{Scriptie: “Are there differences between Flemish pupils \\ acquiring English and pupils with a migratory background \\ living in
    Flanders acquiring English prior to
    instruction?”}}
  \entry
    {2010-2013}
    {Bachelor in het Onderwijs}
    {Arteveldehogeschool Gent}
    {\emph{Secundair onderwijs: Wiskunde - Engels} \\ \textit{Scriptie: “English teacher training in Vietnam: \\ an investigation of the didactics at the
    teacher training \\department and in secondary schools.”}}
  \entry
    {2009-2010}
    {Bachelor of Science in de \\ biomedische wetenschappen}
    {Universiteit Gent}
    {\emph{Onvoltooid} (\textit{discrepantie tussen inhoud en verwachting})}
  \entry
    {2008-2009}
    {7\textsuperscript{e} jaar Secundair na Secundair}
    {Koninklijk Atheneum Gent}
    {Bijzondere wetenschappelijke vorming}
  \entry
    {2002-2008}
    {Secundair onderwijs}
    {Sint-Franciscus Evergem}
    {Economie-moderne talen}
\end{entrylist}

\section{werkervaring}
\vspace{-0.2cm}
\begin{entrylist}
  \vspace{-2.5mm}
  \entry
    {2019 - ...}
    {Leerkracht wiskunde-Engels}
    {GO! Atheneum Gentbrugge}
    {}
  \vspace{-2.5mm}
  \entry
    {2015 - 2020}
    {Leerkracht wiskunde-Engels}
    {GO! Atheneum Merelbeke}
    {}
  \vspace{-2.5mm}
  \entry
    {2014 - 2020}
    {Praktijk assistent: taal- en tekstvaardigheid}
    {Universiteit Gent}
    {}
  \vspace{-2.5mm}
  \entry
    {2013 - 2015}
    {Restaurantwerk}
    {Universiteit Gent}
    {}
  \vspace{-2.5mm}
  \entry
    {2013 - 2014}
    {Stafondersteuning Eenheden en Evenementen}
    {FOS Open Scouting}
    {}
  \vspace{-2.5mm}
  \entry
    {2006 - 2014}
    {Verscheidene vakantiejobs}
    {}
    {}
\end{entrylist}

\section{vrijwilligerswerk}
\vspace{-0.2cm}
\begin{entrylist}
    \entry
    {2018 - ...}
    {Reisbegeleider}
    {KrisKras}
    {duurzame groepsreizen}
    \entry
    {\hspace{-1.5cm}Feb-Apr. 2013 }
    {‘Poussi\`eres de Vie’}
    {Kon Tum, Vietnam}
    {Lesgeven en organiseren van activiteiten in weeshuizen}
    \entry
    {2012 - ...}
    {FOS Open Scouting}
    {Gent en Knokke-Heist}
    {\vspace{-0.25cm}\begin{itemize}
      \item Leiding en kookploeg scoutsvereniging
      \item Assistent-Eenheidsleiding/Hoofdleiding (financi\"en)
      \item Nationaal vrijwilliger (evenementen en medewerkersmanagement)
    \end{itemize}}
\end{entrylist}

\end{document}


  