%!TEX TS-program = xelatex
\documentclass[]{friggeri-cv}
\addbibresource{bib/bibliography.bib}
\hypersetup{    colorlinks=true,
    linkcolor=black,
    citecolor=black,
    filecolor=black,
    urlcolor=black,}
\begin{document}

\header{Bram }{De Jaegher}
       {Bioscience engineer}{Curriculum}{Vitae}
\begin{aside}
  \section{about}
    9000 Gent
    Belgium \vspace{0.21cm}
    +...
    \linkBlock{\href{mailto:bram.de.jaegher@gmail.com}{bram.de.jaegher@gmail.com}}
    \linkBlock{\href{http://be.linkedin.com/in/BramDeJaegher}{LinkedIn}}
    \linkBlock{\href{https://github.com/Beramos}{Github}} \vspace{0.21cm}
    Driver's licence: B \vspace{-1.25mm}
  \section{languages}
    Dutch: native language
    English: C2 (CEFR)
    French:  B1  (CEFR) \vspace{3.3mm}
  \section{additional skills}
  Computational fluid dynamics
  Mathematical modelling
  Machine learning
  Control theory 
  LaTeX \vspace{3.3mm}
  \section{programming} 
  \textit{Working knowledge}
  MATLAB/Simulink
  Python 2/3
  OpenFOAM (CFD) \vspace{0.15cm}
  \textit{Basic knowledge}
  HTML/CSS/JS
  C++
  R
\end{aside}
\vspace{3mm}
\textbf{keywords:} mathematical modelling, process control, computational fluid dynamics, machine learning, optimisation, chemistry, problem solving
\vspace{3mm}

\section{education}
\begin{entrylist}
    \entry
    {2014–2016}
    {M.Sc. summa cum laude}
    {Ghent University}
    {Bioscience engineering\\
    \emph{Chemistry and bioprocess technology}}
  \entry
    {2011-2014}
    {B.Sc. cum laude}
    {Ghent University}
    {Bioscience engineering\\
    \emph{Chemistry and food technology}}
  \entry
    {2005-2011}
    {GCSE in Math and Sciences}
    {Royal Atheneum Knokke-Heist}
    {4.2/5 GPA}
\end{entrylist}

\section{experience}
\begin{entrylist}
  \entry
    {01/2017 - ...}
    {PhD candidate}
    {Ghent University}
    {Model-based optimisation of design and operation of bioreactors with a focus on gas-liquid mass transfer \\ \textit{BIOMATH}}
    \entry
    {09-12/2016}
    {Research assistant}
    {Ghent University}
    {Mathematical modelling of filtercake formation and fungal growth \\{\textit{BIOMATH/KERMIT}}}
    \entry
    {08-09/2014}
    {Research internship}
    {University of S\~{a}o Paulo}
    {Computer vision techniques for polymer recognition\\ using atomic-force microscopy images}
\end{entrylist}

\section{scriptions}
\begin{entrylist}
  \entry
    {2016}
    {Master thesis}
    {Ghent University}
    {Spatio temporal modelling of filtercake formation\\ in filtration processes}
  \entry
    {2014}
    {Bachelor thesis}
    {Ghent University}
    {Innovative applications of artificial intelligence\\ in the food industry}
\end{entrylist}

\section{voluntary work}
\begin{entrylist}
 \entry
    {2009-2014}
    {Leader youth movement}
    {Knokke-Heist}
    {102\textsuperscript{e} FOS De Albatros}
\end{entrylist} \vspace{1.5cm}
\section{projects}
\begin{entrylist}
 \entry
 {2016}
 {Open Webslides}
 {Ghent University}
 {Open-source platform for interactive presentation slides \\ \emph{UGent innoversity challenge winner}}
 \entry
 {2016}
 {Dewpal: biocatalysed atmospheric condenstation}
 {MIT, Boston}
 {iGem: International Genetically Engineered Machine \\Competition}
\end{entrylist}
%%%%% Hyperlinks images %%%%%
\makeatletter
\def\parsecomma#1,#2\endparsecomma{\def\page@x{#1}\def\page@y{#2}}
\tikzdeclarecoordinatesystem{page}{
    \parsecomma#1\endparsecomma
    \pgfpointanchor{current page}{north east}
    % Save the upper right corner
    \pgf@xc=\pgf@x%
    \pgf@yc=\pgf@y%
    % save the lower left corner
    \pgfpointanchor{current page}{south west}
    \pgf@xb=\pgf@x%
    \pgf@yb=\pgf@y%
    % Transform to the correct placement
    \pgfmathparse{(\pgf@xc-\pgf@xb)/2.*\page@x+(\pgf@xc+\pgf@xb)/2.}
    \expandafter\pgf@x\expandafter=\pgfmathresult pt
    \pgfmathparse{(\pgf@yc-\pgf@yb)/2.*\page@y+(\pgf@yc+\pgf@yb)/2.}
    \expandafter\pgf@y\expandafter=\pgfmathresult pt
}
\makeatother

\begin{tikzpicture}[remember picture,overlay ,every node/.style={anchor=center}]
  \node at (page cs:0.66,-0.9) {\small References available on request}; 
\end{tikzpicture}

\end{document}

